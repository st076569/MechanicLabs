%%%%%%%%%%%%%%%%%%%%%%%%%%%%%%%%%%%%%%%%%%%%%%%%%%%%%%%%%%%%%%%%%%%%%%%%%%%%%%%%
% > Лабораторная работа 2: Определение частоты собственных колебаний 
%   вибрационного столика.
% > Баталов Семен, Антонова Мария, Клюшин Максим.
% > 2021 год.
%%%%%%%%%%%%%%%%%%%%%%%%%%%%%%%%%%%%%%%%%%%%%%%%%%%%%%%%%%%%%%%%%%%%%%%%%%%%%%%%

\documentclass[12pt, a4paper]{article}
\usepackage[left=2cm, right=2cm, top=2.5cm, bottom=2.5cm, nohead, 
footskip=1cm]{geometry}
\usepackage{graphicx}
\graphicspath{{./Pictures/}}
\usepackage[utf8]{inputenc}
\usepackage[english, russian]{babel}
\usepackage{indentfirst}
%\usepackage{array}
%\usepackage{longtable}
\usepackage{misccorr}
\usepackage{setspace,amsmath}

\begin{document}
    
    \begin{center}
        \large{Санкт-Петербургский государственный университет} \\
        \large{Saint-Petersburg State University}\\
        \hfill \break
        \hfill \break
        \hfill \break
        \hfill \break
        \hfill \break
        \hfill \break
        \hfill \break
        \large{Кафедра теоретической и прикладной механики} \\
        \hfill \break
        \hfill \break
        \large{\textbf{ОТЧЕТ}} \\
        \large{\textbf{По лабораторной работе 2}} \\
        \large{<<Определение частоты собственных колебаний вибрационного 
            столика>>} \\
        \hfill \break
        \hfill \break
        \hfill \break
        \large{По дисциплине} \\
        \large{<<Лабораторный практикум по теоретической механике>>} \\
    \end{center}
    
    \hfill \break
    \hfill \break
    \hfill \break
    \hfill \break
    \hfill \break
    \hfill \break
    \hfill \break
    
    \begin{flushleft} 
        \large{Выполнили:} \\
        \large{Баталов Семен} \\
        \large{Антонова Мария} \\
        \large{Клюшин Максим} \\
    \end{flushleft}
    
    \hfill \break
    \hfill \break
    \hfill \break
    \hfill \break
    
    \begin{center} 
        \large{Санкт-Петербург} \\
        \large{2021} \\
    \end{center}
    
    \thispagestyle{empty}
    \newpage
    
\end{document}